\documentclass[]{book}

\usepackage{graphicx}
\usepackage{minted}
\usepackage{multicol}
\usepackage[dvipsnames]{xcolor}
\usepackage[colorlinks=true,urlcolor=blue]{hyperref}
\usepackage{geometry}
\geometry{%
  a4paper,
  left=2.2cm,
  right=2.2cm,
  top=2.5cm,
  bottom=2.5cm
}

\usemintedstyle{one-dark}

\begin{document}

\title{Interview Questions}
\author{%
  Snapp Team \\[1cm]
  \includegraphics[width=10em]{./snapp.png}
}

\maketitle

\chapter{Warm-up}

\section{Introduction}

\subsection{Interviewer}
\begin{itemize}
  \item Who I am?
  \item Snapp! was based on \textbf{BaseAPI} in PHP
  \item Snapp! is now based on many Golang/PHP Microservices
  \item Verticals and their members
    \begin{itemize}
      \item Tech Specialist
      \item QA
      \item Backend (4--5 members)
      \item Product
      \item Scrum
      \item Android
      \item PWA
    \end{itemize}
  \item Agile Methology (Sprint, Quater, OKR, \ldots)
  \item Each vertical has its services
    \begin{itemize}
      \item Dispatching
      \item RLC (Ride Life Cycle)
      \item Offering
      \item Driver Ride Experience
      \item Passenger Ride Experience
      \item \ldots
    \end{itemize}
  \item Infrastructure:
    \begin{itemize}
      \item VM (OpenNebula)
      \item Cloud based on Kubernetes (Openshift)
    \end{itemize}
\end{itemize}

\subsection{Interviewee}

\begin{itemize}
  \item Challenges which you are proud of
  \item University
    \begin{itemize}
      \item Major
      \item Courses
    \end{itemize}
  \item Where do you see yourself in the next 5 years?
  \item Why and when did you get promoted?
  \item Why do you think Snapp is for you?
\end{itemize}

\chapter{Trainee/Junior}

\section{Algorithm}

\subsection{Time Complexity}

\begin{itemize}
  \item Time Complexity Definition
\end{itemize}

\subsection{Linked List}

\begin{itemize}
  \item Differences between linked list and arrays
  \item Time complexity for accessing an element
  \item An ArrayList, or dynamically resizing array, allows you to have the benefits of an array while offering flexibility in size.
    How do they do this? consider we want to add n number into ArrayList what is the time complexity?
\end{itemize}

\subsection{Sort Algorithms}

\begin{itemize}
  \item Which sort has the best order among the comparison sorts?
  \item What is the difference between Merge Sort and Quick Sort?
  \item Is there any sorting algorithm that has $O(n)$?
\end{itemize}

\subsection{Greedy Algorithms}

\section{Operating Systems}
\begin{itemize}
  \item Process vs Threads
  \item Multi-thread application programming experience
  \item How we can get the processes list in linux (\textit{ps})
  \item Do you know \textit{grep}?
\end{itemize}

\section{Networking}
\begin{itemize}
  \item What are the differences between TCP and UDP\@?
  \item Flow Control vs Congestion Control
  \item How does a PHP request flow work?
  \item Is there any restriction on the number of TCP connections for a system?
\end{itemize}

\section{Git}
\begin{itemize}
  \item Where did you use git?
  \item Differences between Git and Github
  \item What process is an alternative to merging?
  \item How do you revert a commit that has already been pushed and made public?
  \item Do you remember some of your most used git command?
  \item Rebase vs Merge Differences
\end{itemize}

\section{Database}

\begin{itemize}
  \item Foreign Key
  \item Primary Key
  \item NoSQL vs SQL
\end{itemize}

\section{Docker}

\begin{itemize}
  \item Container vs Virtual Machine
\end{itemize}

\section{SOLID}

\begin{itemize}
  \item \textbf{S}:\@ Single Responsibility Principle (known as SRP)
  \item \textbf{O}:\@ Open/Closed Principle
  \item \textbf{L}:\@ Liskov’s Substitution Principle
  \item \textbf{I}:\@ Interface Segregation Principle
  \item \textbf{D}:\@ Dependency Inversion Principle
\end{itemize}

\section{Go}

\begin{itemize}
  \item Channels/Goroutine
    \begin{itemize}
      \item Buffer/Unbuffered Channels
      \item \mintinline{go}|select|
      \item Can you explain the following cases in Golang \href{https://stackoverflow.com/questions/39015602/how-does-a-non-initialized-channel-behave)}{Answer}:
        \begin{minted}[bgcolor=Black]{go}
var ch1 chan int
// write on a nil channel
ch1 <- 1
// read from a nil channel
<-ch1

ch2 := make(chan int)
close(ch2)
// write on a closed channel
ch2 <- 1
// read from a closed channel
<-ch2
        \end{minted}
      \item Solve Reader-Writer problem with channels
      \item Explain mutex implementation with channels
    \end{itemize}

  \item Embedding
    \begin{minted}[bgcolor=Black]{go}
type Student struct {
  Person
}

type Person struct {
  Name string
  Age  int
}
    \end{minted}

  \item Interfaces
    \begin{itemize}
      \item How they are different from Java interfaces?
    \end{itemize}

  \item Testing
    \begin{itemize}
      \item Have you ever written tests for you Go projects?
      \item Have you ever used \textit{mock} in your projects?
    \end{itemize}

  \item Empty Structure and Why?
    \begin{minted}[bgcolor=Black]{go}
type Empty struct {}
    \end{minted}

  \item Errors
    \begin{itemize}
      \item  Describe the error handling procedure in Go and error wrapping.
      \item What is the context? How we can use it to cancel the long-run processing?
      \item Explain \mintinline{go}{panic} in Golang. Can you mention some of these cases?
    \end{itemize}

  \item Sync Package (Mutex and Semaphore, WaitGroup)
  \item Is there any difference between \textit{\color{YellowOrange} array} and \textit{\color{YellowOrange} slice} in Golang?
  \item Explain the result of the following code. Is there any issue?
    \begin{minted}[bgcolor=Black]{go}
package main

import "fmt"

func main() {
  s := []int{1, 2, 3, 4}
  change(s)
  fmt.Println(s)
}

func change(s []int) {
  t := make([]int, len(s))
  copy(t, s)
  for i := range t {
    t[i]++
  }
  s = t
}
    \end{minted}

\end{itemize}

\section{Kubernetes}

\begin{itemize}
  \item Did you write a kubernetes manifest?
  \item Why we need \textit{service} for accessing to kuberenete pods?
  \item Can we use pod's ip address for getting access to it?
  \item What are the differences between readinees and liveness probes?
  \item Do you know Helm/Kustomize/\ldots?
\end{itemize}

\section{Cloud Native Design}

\begin{itemize}
  \item how do you handle a crashed loop application on kubernetes?
  \item how do you monitor an application?
    \begin{itemize}
      \item Metrics (Telemetry)
      \item Logs
      \item Tracing
    \end{itemize}
\end{itemize}

\section{System Design}

\begin{itemize}
  \item What do you know about deployment
  \item Problems:
    \begin{itemize}
      \item Snapp Event Delivery based on \textit{MQTT}, \textit{HTTP}, \ldots
      \item URL Shortener which contains
        \begin{itemize}
          \item \textit{Redis}
          \item \textit{Database Replication/Sharding}
          \item \textit{HAProxy}
          \item \ldots
        \end{itemize}
      \item Voting System that introduces the \textbf{CAP} theorem
    \end{itemize}
\end{itemize}

\section{Hands-on Problems}

Problems can be written in a shared google document in whatever language Interviewee likes.

\subsection{Shuffle}

You have an array with n-items named $A$.
We want to partition it into $k$-subarrays that each of them has $n/k$ items, and each element of $A$ appears precisely once.
The order of these subarrays must not be the same as the $A$ and also we know that: $n \% k == 0$.
For example:

\begin{minted}[bgcolor=Black]{python}
A = [1, 2, 3, 4]
k = 2
\end{minted}

we don't accept the following solution:

\begin{minted}[bgcolor=Black]{python}
A1 = [1, 2]
A2 = [3, 4]
\end{minted}

but we accept the following solution:

\begin{minted}[bgcolor=Black]{python}
A1 = [1, 3]
A2 = [2, 4]
\end{minted}

\begin{itemize}
  \item with duplication or without duplication?
\end{itemize}

\subsection{Coins}

We have $n$ amount of money and our country have the following coins:

\begin{enumerate}
  \item coin-1
  \item coin-5
  \item coin-7
  \item coin-10
\end{enumerate}

we want to have this money with minimum number of coins.\ what is the minimum number of required coins?
for eaxmple:

\begin{minted}[bgcolor=Black]{python}
2 = 2 * coin_1
5 = 1 * coin_5
6 = 1 * coin_5 + 1 * coin-1
\end{minted}

\subsection{$k$ nearest point}

We have $n$ points and one reference point.
Each point has $x$ and $y$ coordinates.
We want to find $k$ the nearest points to the reference point.
For example:

\begin{minted}[bgcolor=Black]{python}
import dataclasses

@dataclasses.dataclass()
class Point:
  x: float
  y: float

points = [
  Point(0, 0), Point(0, 1), Point(1, 1), Point(1, 0),
  Point(-1, -1), Point(0, -1), Point(-1, 0),
]
reference = Point(-1, -1)
n = len(points)
k = 2

k_nearest_points = [Point(-1, -1), Point(-1, 0)]
# or
k_nearest_points = [Point(-1, -1), Point(0, -1)]
\end{minted}

\chapter{Senior/Manager}

\end{document}
